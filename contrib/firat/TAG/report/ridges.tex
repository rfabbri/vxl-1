\subsection{Ridge and Valley Topographic Curves}
\begin{definition} 
A {\em ridge} or {\em valley} point of $z=f(x,y)$ is a point where
\begin{equation}
\overrightarrow{\grad f}.\overrightarrow{\nu_1} = 0, \label{eq:07-27-06:starstar10}
\end{equation}
where $\nu_1$ is the eigenvector with the largest eigenvalue of the Hessian of $f$, and where the
eigenvalue $\lambda_1>0$ or $\lambda_1<0$, respectively. \todo{[Letting $\xi$ parametrize direction $\overrightarrow{\nu}$, we have]}
\end{definition}

\noindent In analogy to the condition in 1D used to distinguish minima from maxima among the extrema of 1D functions, namely,
whether $f^{''}(x)$ is positive or negative, respectively, ridges and valleys are distinguished using the
second derivative in the direction $\nu_1$,
\begin{equation*}
\frac{\partial^2 f}{\partial\xi^2} = \nu_1^T H \nu_1 
= \nu_1^T(\lambda_1 \nu_1)
= \lambda_1,
\end{equation*}
which gives just the eigenvalue $\lambda_1$.
Thus, if $\lambda_1 < 0$, we have a ridge and if $\lambda_1 > 0$ we have a valley. The case $\lambda_1=0$ also implies $\lambda_2=0$
since $\lambda_1$ is the larger eigenvalue. In such a case higher-order derivatives needs to be consulted to determine a ridge.
